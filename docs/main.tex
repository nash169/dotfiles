 %!TEX program = pdflatex

\documentclass[12pt]{article}

\title{Linux Configuration}
\author{Bernardo Fichera}
\date{\today}

\usepackage{listings}
\usepackage{hyperref}

\begin{document}

\maketitle

\section*{Arch Linux installation}
The \href{https://wiki.archlinux.org/index.php/Installation_guide}{Arch wiki} for the installation of OS seems to be pretty complete. However minor things are missing especially for what concern special settings for installing Arch on Mac.

\subsection*{Flash USB}
For flashing an USB key using \textbf{dd} under linux the command seems to be the following
\begin{lstlisting}[language=bash]
    $ dd bs=4M if=path/to/archlinux.iso 
    of=/dev/sdx status=progress oflag=sync
\end{lstlisting}
However that fails on macOS and the above command has to be substituted with
\begin{lstlisting}[language=bash]
    $ dd bs=4m if=path/to/archlinux.iso 
    of=/dev/sdx; sync
\end{lstlisting}

\subsection*{Setting the font}
Consol mode installation might suffer of microscopic text on high res display. In order to adjust the font use
\begin{lstlisting}[language=bash]
    $ setfont latarcyrheb-sun32 (setfont sun12x22)
\end{lstlisting}

\subsection*{Interrupt issue on Mac}
It seems that there is something wrong with the interrupts on Mac. To find out which one is going crazy run
\begin{lstlisting}[language=bash]
    $ grep . -r /sys/firmware/acpi/interrupts/gpe*
\end{lstlisting}
Disable it
\begin{lstlisting}[language=bash]
    $ echo "disable" > 
    /sys/firmware/acpi/interrupts/gpe06
\end{lstlisting}
This will automatically disable by the bootloader options when configure. Although at the moment the solution is for systemd-boot and not the more common grub. Check how to solve it for grub if the problem persist.

\subsection*{Internet connection}
The easiest solution is ubs tethering. Use either the phone or a portable router. In order to detect the network interfaces run
\begin{lstlisting}[language=bash]
    $ ip link
\end{lstlisting}
To connect to a specific device
\begin{lstlisting}[language=bash]
    $ dhcpcd <your_interface_name>
\end{lstlisting}

\subsection*{Create partitions}
For listing the devices
\begin{lstlisting}[language=bash]
    $ fdisk -l
\end{lstlisting}
and then simply
\begin{lstlisting}[language=bash]
    $ fdisk /dev/sdX
\end{lstlisting}
to configure the device. Everything is pretty clear; just remember to set the correct type for the EFI (and SWAP) partition otherwise it is not possible to install the bootloader. Specifically $1 - 23 - 19$ ($EFI System - Linux root (x86-64) - Linux swap$).

\subsection*{Format partitions}
This are the commands to format all the necessary partitions
\begin{lstlisting}[language=bash]
    $ mkfs.fat -F32 /dev/efi_partition
    $ mkfs.ext4 /dev/root_partition
    $ mkswap /dev/swap_partition
\end{lstlisting}

\subsection*{Mount partitions}
To mount the partitions
\begin{lstlisting}[language=bash]
    $ mount /dev/root_partition /mnt
    $ mkdir -p /mnt/efi && mount /dev/efi_partition /mnt/efi
\end{lstlisting}
and then activate the swap as explained in the Arch wiki.

\subsection*{System installation}
Pretty much the same of the wiki; just add a couple packages
\begin{lstlisting}[language=bash]
    $ pacstrap /mnt base base-devel linux linux-firmware git
\end{lstlisting}
After having generated the /etc/fstab one tutorial suggests to add the following line to the file
\begin{lstlisting}[language=bash]
    rw,relatime,data=ordered,discard
\end{lstlisting}
Not sure how important it is. Follow the wiki for a while. The only addition is to edit \textbf{/etc/vconsole.conf} if you want to keep the font settings with
\begin{lstlisting}[language=bash]
    FONT=latarcyrheb-sun32
\end{lstlisting}

\subsection*{Create user}
If you want to create another user other than root run
\begin{lstlisting}
    $ useradd -m -g users -G wheel -s /bin/bash <USER> && passwd <USER>
\end{lstlisting}

\subsection*{Mac specific modules}
We need to enable some kernel modules for fan speed and temperature sensors, open \textbf{/etc/modules} file and insert two lines:
\begin{lstlisting}
    coretemp
    applesmc
\end{lstlisting}

\subsubsection*{Bootloader installation}
Here I'll report how to install the grub. Although check this \href{https://medium.com/@laurynas.karvelis_95228/install-arch-linux-on-macbook-pro-11-2-retina-install-guide-for-year-2017-2034ceed4cb2}{Install Arch on Mac} in order to see how to setup systemd-boot with the fix for the crazy interrupt.
Install the following pack for microcode update
\begin{lstlisting}
    $ pacman -S  intel-ucode (amd-ucode)
\end{lstlisting}
Most tutorial install also other things
\begin{lstlisting}
    $ pacman -S  dosfstools os-prober mtools
\end{lstlisting}
For sure in order to install the grub you need
\begin{lstlisting}
    $ pacman -S  efibootmgr grub
\end{lstlisting}
Now you can install the grub
\begin{lstlisting}
    $ grub-install --target=x86_64-efi --efi-directory=efi --bootloader-id=grub_uefi
\end{lstlisting}
and generate the configuration file
\begin{lstlisting}
    $ grub-mkconfig -o /boot/grub/grub.cfg
\end{lstlisting}

\subsection*{Fix unwanted laptop resume after lid is closed}
To check run
\begin{lstlisting}
    $ cat /proc/acpi/wakeup
\end{lstlisting}
To fix the problem edit $/etc/udev/rules.d/90-xhc\_sleep.rules$ adding the following lines
\begin{lstlisting}
    # disable wake from S3 on XHC1
    SUBSYSTEM=="pci", KERNEL=="0000:00:14.0", ATTR{power/wakeup}="disabled"
\end{lstlisting}

\subsection*{Network}
If you are using an old mac this are the drivers
\begin{lstlisting}
    $ yay -S broadcom-wl
\end{lstlisting}
and then set network manager
\begin{lstlisting}
    $ sudo systemctl disable dhcpcd
    $ sudo pacman -S networkmanager network-manager-applet
    $ sudo systemctl enable NetworkManager
\end{lstlisting}
After rebooting you should be able to connect with
\begin{lstlisting}
    $ nmcli dev wifi connect <wifi-name> password <password>
\end{lstlisting}

\section*{Raspberry}
\begin{lstlisting}
    raspi-config
    python-raspberry-gpio
    python-smbus-git
    i2c-tools
\end{lstlisting}
\end{document}